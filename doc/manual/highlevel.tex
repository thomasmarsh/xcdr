\chapter{Highlevel Architecture Description}

\begin{intro}
The first part of this chapter presents a short overview
of the philosophy of \XCDR{}. The second part
focuses on the basic elements involved in the \XCDR{} system.
After reading this chapter, you should have a rough knowledge
of how \XCDR{} works, which you will need to understand the rest
of this book.
\end{intro}

\section{Overview of \XCDR{}}

	\XCDR{} is the suite of applications which define a generic mediation and
	billing system suitable for anything from the small ISP who wishes to charge,
	for example, users for their disk usage, to the telephony wholesale carrier
	with a worldwide network. Due to its modular design (based upon a series of
	generators providing information to the next process in line) it has proven
	to be at least ten times faster (1000 cps) on inexpensive
	hardware.\footnote{The test system at the time of initial development was a
		PC running Linux. The CPU was an Athlon-XP 1.6 GHz, 256 Mb RAM
		was available, and a standard IDE controller was used (i.e., no
		SCSI or RAID which yield much better system performance). The
		system was inserting billing records into a MySQL v4.0.13
		database.}

	The basic tenets of the \XCDR{} architecture are:

	\begin{itemize}
	  \item Process data as soon as it is available
	  \item Each functional processing unit should perform one task, and that task should be performed well (a UNIX tenet)
	  \item Doing a job well means: fast, simple, distributable across the network, and easily configurable
	\end{itemize}

	To meet its design goals, \XCDR{} defines the following operational contexts:

	\begin{itemize}
	  \item Collection
	  \item Authorization
	  \item Routing
	  \item Mediation
	  \item Rating
	  \item Error Handling
	  \item Monitoring
	  \item Reporting
	\end{itemize}

	Further operational contexts that may be implemented in the future are:
	\begin{itemize}
	  \item Invoicing
	  \item Taxation
	\end{itemize}

	For now, it is intended that 3rd party products be used for these purposes.
%	<!--em>TODO: mention Oracle Financials, QuickBooks, etc.. here.</em-->

	With the combination of database replication the various components of the
	system, the process of billing activity on the network can be split across
	multiple database or machine boundaries providing for arbitrary scalability.
	You can place collection points and mediators close to the source
	(networkologically close to the gateway) to avoid bandwidth consumption at
	this point. The records may be forwarded to a single database for billing,
	or even the rating instance may be run simultaneously on multiple machines,
	providing that a round-robinning scheme is specified.


	\index{collection}\section{Collection}
		The point of entry of any service activity is via a collection point, or
		a \index{Reader}\emph{Reader}. A simple example of a \emph{Reader} is the
		\index{LocalFileReader}\emph{LocalFileReader} which reads individual
		lines from a file (each
		line indicating accounting for a type of service) and serves as the entry
		point to the billing process. Another, more complicated, collection
		service might be bundled with Authorization, as in a RADIUS, or SIP
		server.

	\index{authorization}\section{Authorization}
		Like collection, authorization is another entry point into the billing
		system for external services. An authorization consists of actions such
		as determining whether a customer has exceeded his credit limit, or if
		a prepaid caller has any balance left. In general, this service requires
		a set of inputs, and based upon site implementation rules, allows or
		denies the user to continue with a certain action.

		A concrete example of an authorization service would be an prepaid calling
		card IVR system which authorizes via RADIUS. The customer dials an access
		number, enters their PIN, hears a statement about their account (``You have
		fifteen dollars and thirty-two cents remaining''), enters a destination
		number, is connected and talks, and is disconnected by the system when
		their time limit is exceeded. From the authorization standpoint, the
		system actions are:
		\begin{itemize}
		  \item check that the account number is valid
		  \item return a balance, currency, and preferred language (optional) to the IVR server
		  \item calculate the maximum amount of time the user may speak for the provided destination (actual, and announced)
		  \item if any of the above operations fails, send a REJ (reject) packet back to the IVR server.
		\end{itemize}

	\index{routing}\section{Routing}
		In the telephony environment an expectation of the billing
		system is that it can act as a gatekeeper. This is due to the fact that
		the billing system contains:
		\begin{itemize}
		  \item knowledge of all billable gateways,
		  \item knowledge of routes between gateways,
		  \item the costs associated with certain vendors/routes,
		  \item the current Average Success Ratio (ASR) for each route, and
		  \item the capacity of each route.
		\end{itemize}

		Using the above information plus user inputs (such as ``round-robin
		percentages''), the billing system can serve dynamic routes to gateways
		using customer specific criteria.

	\index{mediation}\section{Mediation}
		The mediation service is completely internal to the operation of the
		billing system. It is affected by site specific implementation, but is
		not exposed to external services utilizing the billing system. Mediation
		is the process of translating collected records into Service Detail
		Records (SDRs), or, in other words, a uniform internal representation
		that can be handled by the billing system for purposes of rating and
		reports.

		For every type of input, a mediation service is required. An example
		would be a mediator for \VOIP gateway which produces a text Call Detail
		Record (CDR) file. The CDRs are in a vendor specific format and may
		require that multiple records be grouped together before a service
		usage may be billed, as in the case where the gateway produces START, STOP,
		and INTERMEDIATE records. Together, the mediator may perform the following
		actions (in some cases as a service to the collection process):

		\begin{itemize}
		  \item provide a unique identifier for a usage
		  \item determine that a usage has completed (or will not complete), and
		        begin the conversion to an SDR
		  \item associate an activity type to the usage
		  \item determine any variables relevant to billing (e.g., start time, stop
		        time, time zone, duration, if a call is from a pay phone)
		  
		  \item insert an SDR into the processing stream, and clear the usage
			from the collection records
		\end{itemize}

		Obviously, the mediator does a lot, seeming to break one of the
		architectural tenets, but they actually tend to be very simple.
		You may read more about the implementation of mediators in section XXX.

	\section{Rating}
		By far, the majority of the interaction with the billing system consists
		of maintaining rating information. \XCDR{} tries to achieve flexibility
		without sacrificing efficiency. To this end, the \emph{Rater} tends
		to operate like an in-memory database, highly specific to a specific
		task. The user interaction with the \emph{Rater} consists of:
		\begin{itemize}
		  \item defining customers (revenues) and vendors (costs);
		  \item specifying the means of identification of customers and vendors (e.g., gateway IP address, egress trunk group, PIN number);
		  \item providing lists of products, tariffs, and rates;
		  \item specification of other rating parameters (e.g., surcharges, flat rates, discounts, duration base intervals (rounding) and brackets, holiday rates, etc.)
		\end{itemize}

		The rater tends to be quite fast (in excess of 14,000 rates applied per
		second on test systems), but the main concern is actually one of
		administration: firstly, some administrative staff member has to maintain
		large amounts of data to be utilized by the system; secondly, as there may
		be multiple rating instances in operation in the network, \XCDR{} must
		smartly update all rating instances with updated information (new rates,
		new customer, canceled rates, etc.)

	\section{Error Handling}
		The number of components involved in the mediation process require
		an effective means of handling error conditions. When a \emph{Reader} is
		unable to read, the chain of processes is broken from that source. Coupled
		with \emph{Monitoring} (the next topic), errors at any point should be
		flagged with detail appropriate to the context. For example, when rating
		a telephone call it is necessary to identify which time zone the caller
		is coming from. If this cannot be done, then the activity should be marked
		with the appropriate error code, and the detail necessary to correct
		the situation.

		Another task necessary when handling an error is to move it out of the
		main processing stream. Each generator reads from a queue of pending
		entries. The larger this queue, the slower the read operations will be
		on those queues. As such, a large number of stale records with exception
		flags would result in a massive slowdown should they remain (and further
		accumulate) in the main processing stream.

		Further, as an optimization, each component will, if possible, remember
		the types of exceptions it has already encountered, and prevent undue
		processing of a subsequent record which has the same issue.

	\section{Monitoring}
		A critical aspect of interaction with the billing system is monitoring.
		The users need to keep track of account balances, error conditions,
		quality of service, and possible fraudulent usage. As well, errors
		encountered by the billing system may indicate adverse conditions
		in the network that may not be detected by traditional monitoring
		mechanisms.

		\XCDR{} modules do not directly push information to users
		regarding error conditions. Instead, \XCDR{} provides, internally, a
		system of counters and queues for keeping track of errors. At regular
		intervals, these may be polled for certain criteria.

		The preferred method of monitoring \XCDR{} is via \index{SNMP}SNMP.
		Third-party products such as HP OpenView can import the \XCDR{} MIB
		file and, from a consolidated standard interface, the administrator may
		monitor the billing system in the same way as other network components.

		For environments where SNMP is overkill, \XCDR{} provides a few,
		simpler, alternatives. Users, may view the current status of \XCDR{}
		by going to the administrative web console. To receive alerts in an
		automated fashion, an email delivery system is also provided. Email
		notification is the only push mechanism provided with \XCDR{}. If the
		user requires other mechanisms (pager notification, SMS, a telephone
		call), this should be implemented through an email gateway.  Of course,
		most SNMP based tools already support diverse notification methods.

	\section{Reporting}
		A wealth of information is available about the status of business
		operations through the billing system. The types of
		reports one usually wishes to see are:

		\begin{itemize}
		\item ASR reports showing the number of successful operations as compared
		      to failures,
		\item cost/revenue reports indicating the profitability of provided
		      services,
		\item usage profiles describing which products are highly used and
		      in what ways, and
		\item error reports showing billing errors which need to be corrected
		      by the administrator.
		\end{itemize}

		For each of the reports, there is the need to ``drill down'' from a
		high level view into detailed views. This process of descent into detail
		must be available via many paths so that the administrator can see
		different groupings of data. Some examples of detailed views in a
		telephony environment include:

		\begin{itemize}
		\item grouping by gateway, showing statistics only for a single node
		      on the network;
		\item grouping by origination/destination, in which totals are shown for
		      each country or other administrative region (tariff zone, city or
		      province); and
		\item grouping by customer or vendor, so that totals are shown only for
		      a particular billing entity.
		\end{itemize}

\section{Where to Go From Here}
This chapter has introduced the functional areas of \XCDR{}. Each is treated in more
detail in subsequent chapters.
\endinput
