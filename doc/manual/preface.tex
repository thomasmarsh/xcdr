\chapter{Preface}

\section{What is \XCDR{}?}

\XCDR{} is a collection, mediation and billing system designed for the ISP, VPN provider,
ASP, telephony/\VOIP{} carrier, or any other service provider who needs to perform
highly specific business logic in order to generate revenue from their service usage.

\section{History}
	\XCDR{} is an implementation a call mediation engine used for carrier
        telcos. \XCDR{} was written based on experience constructing a similar
        system at a major wholesale VOIP carrier. That system, CDRParse, supports
        arbitrary scalable call volumes by
	separation of collection, mediation and billing across multiple machines
	for wholesale \VOIP traffic; as well, it supports the North American telephony
	billing model, taking into account tariff zones (LATA), complex requirements
	from other vendors and customers, toll free traffic (billing the recipient),
	and other parameters such as pay phone and prison calls. CDRParse was built
	with the goal of achieving billing at a rate of 100 calls per second (cps) per
	mediation server.

\section{The Documentation}
This manual is designed to provide an introduction to \XCDR{}. It is requisite to read
volume prior to the others as it provides a grouning in the terminology and ideology
of the system. The documentation of \XCDR{} consists of the following manuals:

\begin{description}
\item[Concepts Guide] --
	this manual; discusses \XCDR{} from a high level, and forms the foundation
	for reading the other volumes.

\item[User Guide] --
	describes \XCDR{} usage from the standpoint of the user of the web interface,
	including rate management and provisioning.

\item[Administrator Guide] -- 
	describes low level administration of an \XCDR{} implementation, including
	debugging, sizing, and deployment.
			
\item[Development Guide] --
	describes how to implement custom components based upon the \XCDR{} SDK.

\end{description}

	\subsection{Where to get the latest version of this guide}
	The hard copy of this guide is updated at major releases only and does
	not always contain the latest material for enhancements occuring between
	interrim updates. The online copy of this guide is always up-to-date and
	integrates the latest changes to the product. You can access the latest copy
	of this guide at http://www.tdmi.com/support/doc/.

	\subsection{Conventions}
	This publication uses the following typographical conventions:
	\begin{itemize}
	\item Commands and keywords are \emph{emphasized}
	\item Terminal sessions, console screens, and system file names are displayed
	      in a \texttt{fixed width font}
	\end{itemize}
	In addition, 

