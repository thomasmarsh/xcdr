\chapter{Collection}

\begin{intro}
Every device that you deploy has some mechanism for acquiring information about
its usage. The method of retrieving that information can take many forms. This chapter
discuss the means of enabling billing of devices utilizing various retrieval methods.
\end{intro}

\section{The \Reader{}}
The \Reader{} is a generic name for any collection service. \Readers{}
can read lines out of a local file, they might be required to contact a device
and pull information, or they might be embedded in a more complex protocol such
as RADIUS to enable other mechanisms such as authorization in conjunction with
accounting.  A \Reader{} must be associated with a \Mediator{}, which is the next
processor in the stream.

There are a variety of \Reader{} types provided with \XCDR{}. Some of these are
generic, but many are specific to certain types of devices. The current built-in
\Reader{} types are as follows:

\begin{itemize}
\item \emph{LocalFileReader}
\item \emph{QuintumCDRServer}
\item \emph{CiscoCDRServer}
\item \emph{RadiusReader}
\item \emph{SIPReader}
\end{itemize}

The \Readers{} have no concept of the content of the data they process; instead, they
are only aware of the protocol for acquiring a record. Key in \Reader{} operation is
how it initializes itself. This initialization may require logging in to a device,
forwarding to a particular position in a file, or a number of other parameters. Some
of these operations may require configuration, and the console reflects this when
deploying \Readers{} on the network. The universal logic of every \Reader{} is
described in algorithm~\vref{readerlogicflow}:

\begin{algorithm}[H]
  \SetLine
  \dontprintsemicolon
%  \AlgData{this text}
%  \AlgResult{how to write}
  
  initialization\;
  \While{no shutdown called}{
    read current record\;
    use \Mediator{} to classify record\;
    write record to internal buffer\;
    \If{threshold reached}{
      place contents of buffer in processing queue\;
    }{}
    \If{\Mediator{} is backlogged}{
      idle until \Mediator{} catches up\;
    }
  }
  \caption{Reader Logic Flow\label{readerlogicflow}}
\end{algorithm}

Three things notable about this process: records are \emph{buffered} to increase
thoroughput; \emph{thresholds} are used to flush the buffer at regular intervals
or when the buffer is full; and the \Reader{} will wait for the \Mediator{} to
catch up before processing more records in order to prevent processing queues from
becoming over-full.

\begin{figure}[htbp]
\hspace*{0.75in}
\epsfig{figure=test,width=1in}
\caption{this is an image}
\label{fig:testing}
\end{figure}

\section{Interaction with the \Mediator{}}

Individual readers are created for each source file and all records therein
are dumped directly to the CDR\_RAW table where they are picked up for
processing.

Readers are not specialized by GW manufacturer type (Sonus, MVAM, Cisco,
Quintum, etc.) since they are only responsible for forwarding accounting
traffic. They are however specialized by operation mode, and will mark
the CDR type as coming from a particular GW and of a certain manufacturer
type.


\section{FileReaader}
If the file will exist on the file system (for instance as a NFS mounted
file, or a file created by a local CDR server), then the reader will read
each line as it appears and insert it into the table.

\section{CDRServer}
The CDRServer reader is specialized for Quintum and will listen on a
specified port for incoming records and send ACK's on successful insertion.

\section{RADIUSReader}
The RADIUSReader will use XTRadius to receive Accounting records, insert
them and send an ACK on successful insertion. Any other type of RADIUS
traffic (Authorization and Authentication) is unrecognized at this point.

\section{Mediator}
The Mediators are responsible for retrieving records from CDR\_RAW and
parsing them into an internal representation which will be stored in
the CDR table. The CDR is at this point unrated and should be picked
up by the mediation engine.

Records that cannot be parsed will be placed in the UNRESOLVABLES table.

The Mediators are specialized by GW manufacturer:

\begin{itemize}
  \item Sonus GSX
  \item Lucent MVAM
  \item Siemens DCO
  \item Quintum (Tenor and CMS)
  \item Cisco (AS5400 and equivalent)
  \item Other? (Mera, Vocaltec, Clarnet)
\end{itemize}

The Mediators are responsible for selecting the appropriate records
for processing and performing any de-duplication if necessary. MVAM
in particular is notorious for unusual conditions. On the DCO, the
primary and secondaries need to merged.

\section{Engine}
The billing engine is responsible for mediating the CDR into a billable
format. The billing process includes a number of lookups into various
tables. Ultimately the matching is done by

%	Customer = LookupCustomer( ip )

%	(Call Type, Origination, Destination) -> Rate 

%	Price = Rate * Billable\_Duration

Billed records will be placed in the SDR (Service Detail Record) table.
Records missing critical information will be placed in the exceptions
table.

Call Type is a combination of the following cases:

\begin{itemize}
  \item tollfree / charge (800/888/877)
  \item inter-/intra-state
  \item inter-/intra-LATA
  \item RBOC/Non-RBOC
  \item local/international
  \item payphone (\$0.35 surcharge)
  \item prison
  \item information (555-1212) / out of state information
\end{itemize}

For example, in the case of tollfree, the Origination number will be
used as the Destination for rate determination on the basis of the
rates provided for the customer providing tollfree traffic which
passes through this gateway.

\section{Performance}
One of the primary concerns of a mediation/billing system is simply how many
records can be handled per second. On a relatively meager system which is
bound by having only 256 MB of RAM the system has shown in early phases to
reach speeds of over 1000 calls per second. This means that it is handling
nearly twice that number of records per second as you generally see START/STOP
record combinations forming a single call.

The image below shows mediation of around 75,000 calls in under two minutes.
%<inlinemediaobject>
%<imageobject> <imagedata fileref="graphs/med.png"> </imageobject>
%<imageobject> <imagedata fileref="graphs/med.ps"> </imageobject>
%</inlinemediaobject>
Along the X-axis is the number of seconds since the start of processing. The
Y-axis shows the number of records.

The next graph shows the calls per second of the system for the same run.
%<inlinemediaobject>
%<imageobject> <imagedata fileref="graphs/cps.png"> </imageobject>
%<imageobject> <imagedata fileref="graphs/cps.ps"> </imageobject>
%</inlinemediaobject>
%</para>
%</sect1>
%<!--

\endinput

